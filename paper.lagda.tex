\documentclass{article}
\usepackage{hyperref}
\bibliographystyle{splncs04}
%\usepackage{ucs}
\usepackage[utf8]{inputenc}
\usepackage{amssymb}
\usepackage{amsmath}
\usepackage{txfonts}
%\usepackage{bbm}
%\usepackage[greek,english]{babel}

% agda.sty wants to use the deprecated utf8x option, which
% many publishers don't like. So we do it ourselves

\usepackage{agda}

\usepackage{proof}

% this stuff is related to the Agda latex backend with inputenc/utf8
% not having heard of many charactersrs...

\usepackage{newunicodechar}
\newunicodechar{∋}{$\ni$}
% \newunicodechar{·}{$\cdot$}
\newunicodechar{⊢}{$\vdash$}
\newunicodechar{⋆}{${}^\star$}
\newunicodechar{Π}{$\Pi$}
\newunicodechar{⇒}{$\Rightarrow$}
\newunicodechar{ƛ}{$\lambdabar$}
\newunicodechar{∅}{$\emptyset$}
\newunicodechar{∀}{$\forall$}
\newunicodechar{ϕ}{$\Phi$}
\newunicodechar{ψ}{$\Psi$}
\newunicodechar{ρ}{$\rho$}
\newunicodechar{α}{$\alpha$}
\newunicodechar{β}{$\beta$}
\newunicodechar{μ}{$\mu$}
\newunicodechar{σ}{$\sigma$}
\newunicodechar{≡}{$\equiv$}
\newunicodechar{Γ}{$\Gamma$}
\newunicodechar{∥}{$\parallel$}
\newunicodechar{Λ}{$\Lambda$}
\newunicodechar{₂}{$~_2$}
\newunicodechar{θ}{$\theta$}
\newunicodechar{Θ}{$\Theta$}
\newunicodechar{∘}{$\circ$}
\newunicodechar{Δ}{$\Delta$}
\newunicodechar{λ}{$\lambda$}
\newunicodechar{⊧}{$\models$}
\newunicodechar{⊎}{$\uplus$}
\newunicodechar{η}{$\eta$}
\newunicodechar{⊥}{$\bot$}
\newunicodechar{Σ}{$\Sigma$}
\newunicodechar{ξ}{$\xi$}
\newunicodechar{₁}{$1$}
\newunicodechar{ℕ}{$\mathbb{N}$}
\newunicodechar{ᶜ}{${}^c$}
\newunicodechar{Φ}{$\Phi$}
\newunicodechar{Ψ}{$\Psi$}
\newunicodechar{⊤}{$\top$}
\newunicodechar{≐}{$\doteq$}

% \newunicodechar{×}{$\times$}
\newunicodechar{ℓ}{\ensuremath{\mathnormal{\ell}}}
\newunicodechar{∣}{\ensuremath{\mathnormal{\|}}}
% \newunicodechar{∥}{\ensuremath{\mathnormal{\||}}}
\newunicodechar{₀}{\ensuremath{{}_0}}
% \newunicodechar{₁}{\ensuremath{{}_1}}
\newunicodechar{ℚ}{\ensuremath{\mathnormal{\mathbb{Q}}}}
\newunicodechar{ℤ}{\ensuremath{\mathnormal{\mathbb{Z}}}}
\newunicodechar{∷}{\ensuremath{::}}



\usepackage{hyperref}
\usepackage{cleveref}

\author{Guilherme Horta Alvares da Silva}
% \email{guilhermehas@hotmail.com}
\title{Making all equalities equal}

\begin{document}

\AgdaHide{
\begin{code}
{-# OPTIONS --cubical --type-in-type #-}

module paper where

-- open import Cubical.Foundations.Prelude
-- open import Cubical.Foundations.Isomorphism
-- open import Cubical.Foundations.Equiv
-- open import Cubical.Foundations.Univalence
-- open import Agda.Primitive
-- open import Cubical.Data.Equality
--
open import Agda.Primitive.Cubical
\end{code}
}


\maketitle

\begin{abstract}

\end{abstract}


\section{Propositional Equality}

In the begin of Agda and in most theorems proves, the equality is given by Martin-Löf's definition:

\begin{code}
module Martin-Löf {ℓ} {A : Set ℓ} where

  data _≡_ (x : A) : A → Set ℓ where
    refl : x ≡ x
\end{code}

This equality is very covenient in proof assistances like Agda, because it is possible to pattern match using them:

\begin{code}
  private variable
    x y z : A

  sym  : x ≡ y → y ≡ x
  sym refl = refl

  trans : x ≡ y → y ≡ z → x ≡ z
  trans refl refl = refl
\end{code}

But the problem of this equality is that it does not handle extensionality and other axioms very well.

\begin{code}
module FunExt {ℓ ℓ'} {A : Set ℓ} {B : Set ℓ'} where
  open Martin-Löf

  funExt-Type = {f g : A → B}
    → ((x : A) → f x ≡ g x) → f ≡ g
\end{code}

\section{Cubical Equality}

To solve this problem, Agda adopted cubical type theory that equality is a function from path to types:

\begin{code}
module CubicalEquality {ℓ} {A : Set ℓ} where
  postulate
    PathP : (A : I → Set ℓ) → A i0 → A i1 → Set ℓ

  _≡_ : A → A → Set ℓ
  _≡_ = PathP λ _ → A
\end{code}

From this equality, I will define reflection, symmetry and extensionality:

\begin{code}
module CubicalResults {ℓ ℓ'} {A : Set ℓ} {B : Set ℓ'} where
  open import Cubical.Core.Primitives

  private variable
    x y z : A

  refl : x ≡ x
  refl {x = x} = λ _ → x

  sym : x ≡ y → y ≡ x
  sym p i = p (~ i)

  funExt : {f g : A → B}
    → ((x : A) → f x ≡ g x) → f ≡ g
  funExt p i x = p x i
\end{code}

The operator TODO-invert revert the interval. If the interval \AgdaBound{i} goes from
\AgdaBound{i0} to \AgdaBound{i1}, the interval TODO-invert-i goes from \AgdaBound{i1} to \AgdaBound{i0}.

\section{Leibniz equality}

Leibniz equality is defined in this way:
If \AgdaBound{a} is equal to \AgdaBound{b}, then for every propositional \AgdaBound{P}, if \AgdaBound{P a},
then \AgdaBound{P b}.
The main idea is that if both values are equal, then they are seen equal for every angle.

\begin{code}
module Leibniz {ℓ} {ℓ'} {A : Set ℓ} where
  _≐_ : (a b : A) → Set
  a ≐ b = (P : A → Set ℓ') → P a → P b
\end{code}


%\subsection{}



\section*{Acknowledgements}


\bibliography{bibliography}
\end{document}
